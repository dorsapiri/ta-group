
%
% یک نویسه «%» سبب می‌شود TeX از تمام متن باقیمانده در خط صرف‌نظر کند،
% و نیز برای توضیحهایی مثل این کاربرد دارد.

\documentclass{article}
\usepackage{url}
%\usepackage{fancyvrb}
\usepackage{setspace}\doublespacing
\usepackage{graphicx} 
\usepackage{amssymb}
\usepackage{xepersian}
%\input{C:/Documents and Settings/hawram/Desktop/New Folder (5)/quiz922/first/RO2/G2886EA178-B648-4840-A21A-%AFFADE5AD3E5.ftx_farsitex_cmds_xepersian.tex}

\usepackage{amssymb}
\setlength{\textwidth}{16cm} 
\setlength{\textheight}{26cm}
\addtolength{\oddsidemargin}{-20mm}
\addtolength{\voffset}{-1.4in}
\setlength{\topmargin}{1.cm}
\def \d{\displaystyle}
\def \R{\Bbb{R}}
\def\iiint{\int\!\!\int\!\!\int}

\begin{document}
%\VerbatimFootnotes  
{\bf 
\begin{center}          % پایان دیباچه و شروع متن.
{ \huge به نام خالق یکتا}

{ \Large دانشگاه صنعتی اصفهان-دانشکده علوم ریاضی}

\vspace{0.5cm}
\hrule height .5mm
\vspace{0.5cm}

{\large       
کوئیز اول درس ریاضی عمومی ۱     
    \hspace{0.7cm} 
    ۱۸ اسفند ماه ۹۷
    \hspace{0.7cm}
  مدت ۵۰ دقیقه
  \lr{(G1)}}
  
\vspace{0.5cm}
\hrule height .5mm
\vspace{0.5cm}

{\large نام استاد :
 ......................}
\vspace{0.5cm}

{\large نام و نام خانوادگی : 
................................}
{\large شماره‌ی  دانشجویی 
  : ................................}
\end{center}

\vspace{5mm}

\hrule height .5mm
\vskip 2mm

\noindent {\Large تذکر.}
 این برگه دورو است. اگر به اطلاعات خواسته شده بالا پاسخ ندهید یا نادرست پاسخ دهید کوییز شما تصحیح
 نخواهد شد. چون فقط پاسخ‌نامه تصحیح می‌شود، پاسخ مناسب پرسش‌های داده شده را فقط در پاسخ‌نامه درج کنید.

\vskip 2mm
\hrule height .5mm
\vspace{0.3cm}

\begin{enumerate}
\item
کدام مورد زیر درست است؟
\\
الف) این مورد درست است. 
 \\  
  ب) این مورد درست است.
  \\
ج) این مورد درست است.
 \\ 
  د) این مورد درست است.

\centerline{\rule{5cm}{1pt}}

\item
کدام مورد زیر درست است؟
 \\
الف)‌ این مورد درست است.
\\
ب) این مورد درست است.
\\
 ج) این مورد درست است.
\\
د) این مورد درست است.

\centerline{\rule{5cm}{1pt}}


\item
کدام مورد زیر درست است؟
 \\
الف)‌ این مورد درست است.
\\
ب) این مورد درست است.
\\
 ج) این مورد درست است.
\\
د) این مورد درست است.

\centerline{\rule{5cm}{1pt}}

\item
کدام مورد زیر درست است؟
 \\
الف)‌ این مورد درست است.
\\
ب) این مورد درست است.
\\
 ج) این مورد درست است.
\\
د) این مورد درست است.

\centerline{\rule{5cm}{1pt}}

\item
کدام مورد زیر درست است؟
 \\
الف)‌ این مورد درست است.
\\
ب) این مورد درست است.
\\
 ج) این مورد درست است.
\\
د) این مورد درست است.



\end{enumerate}
\centerline{\rule{5cm}{1pt}}

\begin{center}
{\Large پاسـخ نامـه}\\
\begin{tabular}{|c|c|c|c|c|c|c|c|c|c|c|}
\hline
    & ۱۰  & ۹  & ۸  & ۷  & ۶  & ۵  & ۴  & ۳  & ۲  & ۱  \\ \hline
الف &    &    &    &    &    &    &    &    &    &    \\ \hline
ب   &   &    &    &    &    &    &     &    &     &    \\ \hline
ج   &   &   &   &   &   &   &   &    &   &    \\ \hline
د   &   &   &   &   &   &   &   & &   &    \\ \hline
\end{tabular}
\end{center}

%\vspace{0.1cm}
%\hrule height .5mm
%\vspace{0.1cm}
 
\vspace*{2mm}

\centerline{ {\Large\bf موفق باشید}}
}
\end{document}

